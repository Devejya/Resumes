%!TEX TS-program = xelatex
%!TEX encoding = UTF-8 Unicode
% Awesome CV LaTeX Template for Cover Letter
%
% This template has been downloaded from:
% https://github.com/posquit0/Awesome-CV
%
% Authors:
% Claud D. Park <posquit0.bj@gmail.com>
% Lars Richter <mail@ayeks.de>
%
% Template license:
% CC BY-SA 4.0 (https://creativecommons.org/licenses/by-sa/4.0/)
%


%-------------------------------------------------------------------------------
% CONFIGURATIONS
%-------------------------------------------------------------------------------
% A4 paper size by default, use 'letterpaper' for US letter
\documentclass[11pt, a4paper]{awesome-cv}

% Configure page margins with geometry
\geometry{left=1.4cm, top=.8cm, right=1.4cm, bottom=1.8cm, footskip=.5cm}

% Specify the location of the included fonts
\fontdir[fonts/]

% Color for highlights
% Awesome Colors: awesome-emerald, awesome-skyblue, awesome-red, awesome-pink, awesome-orange
%                 awesome-nephritis, awesome-concrete, awesome-darknight
\colorlet{awesome}{awesome-red}
% Uncomment if you would like to specify your own color
% \definecolor{awesome}{HTML}{CA63A8}

% Colors for text
% Uncomment if you would like to specify your own color
% \definecolor{darktext}{HTML}{414141}
% \definecolor{text}{HTML}{333333}
% \definecolor{graytext}{HTML}{5D5D5D}
% \definecolor{lighttext}{HTML}{999999}

% Set false if you don't want to highlight section with awesome color
\setbool{acvSectionColorHighlight}{true}

% If you would like to change the social information separator from a pipe (|) to something else
\renewcommand{\acvHeaderSocialSep}{\quad\textbar\quad}


%-------------------------------------------------------------------------------
%	PERSONAL INFORMATION
%	Comment any of the lines below if they are not required
%-------------------------------------------------------------------------------
% Available options: circle|rectangle,edge/noedge,left/right
\photo[circle,noedge,left]{profile}
\name{Devejya}{Raghuvanshi}
\position{Data Scientist{\enskip\cdotp\enskip}UWaterloo}
\mobile{(+1) 226-505-5665}
\email{draghuva@uwaterloo.com}
\github{devejya}
\linkedin{draghuva}
% \gitlab{gitlab-id}
% \stackoverflow{SO-id}{SO-name}
% \twitter{@twit}
% \skype{skype-id}
% \reddit{reddit-id}
% \extrainfo{extra informations}


%-------------------------------------------------------------------------------
%	LETTER INFORMATION
%	All of the below lines must be filled out
%-------------------------------------------------------------------------------
% The company being applied to
\recipient
  {Company Recruitment Team}
  {CIBC}
% The date on the letter, default is the date of compilation
\letterdate{\today}
% The title of the letter
\lettertitle{Job Application for Data Science Intern - Fall 2020}
% How the letter is opened
\letteropening{Dear Recruiter,}
% How the letter is closed
\letterclosing{Sincerely,}
% Any enclosures with the letter
\letterenclosure[Attached]{Curriculum Vitae}


%-------------------------------------------------------------------------------
\begin{document}

% Print the header with above personal informations
% Give optional argument to change alignment(C: center, L: left, R: right)
\makecvheader[R]

% Print the footer with 3 arguments(<left>, <center>, <right>)
% Leave any of these blank if they are not needed
\makecvfooter
  {\today}
  {Devejya Raghuvanshi~~~·~~~Cover Letter}
  {}

% Print the title with above letter informations
\makelettertitle

%-------------------------------------------------------------------------------
%	LETTER CONTENT
%-------------------------------------------------------------------------------
\begin{cvletter}

\lettersection{About Me}
I'm a third year undergraduate student at the University of Waterloo, studying Honours Physics and Computer Science. I've worked as a Data Scientist Intern at TD Bank as well as Scotiabank.

\lettersection{Why Me?}
In January of 2017, I started working as a Data Analyst at the Centre for Pattern Analysis and Machine Intelligence. As a Data Analyst, big part of my job was to analyse data, clean, validate it and prepare it for time series analysis.

The next step after preparing the data was to send the data over to other people on the team for the next step. This lead to the obvious issue of not being able to version control the data and retrieve the desired version easily. Dropbox was a way to solve this issue, but would've costed the team ~7000 CAD annually. Unfortunately, that was not affordable option. Therefore, I thought I could try to build an internal web app which allows us to store and manage data. The idea was farfetched, since I had never done any serious web development. I knew python pretty well and some basics of HTML/CSS. However, by the end of the 4 month internship, I was able to build a Dockerised Flask Web App for the whole team where they could login, store and retrieve data and log all that information. I have sample code up on my GitHub (https://github.com/Devejya/UWDataWeb), you can check it out! 

Next internship was at Scotiabank working in the Data Science & Analytics Lab. Our team was working on a web app for Customer Relationship Managers (CRMs) where they will be recommended which Scotia Product to sell to which of their clients. When I got there, my friend (another coop on the team) and I quickly realised after talking with some CRMs, that it was not sufficient to just recommend them a product and client. We need to sell them the recommendation so that they can sell it to the customer. Thus, we need to answer the question, why is this product being recommended for this client.

This prompted me to learn more about ways of doing this for the complex ML models we were using, and I stumbled upon SHAP. I had never worked in that field before, I was able to use SHAP to identify the most important features for the each weekly recommendation and display it in a pretty and effective way on the web app. After several iterations, we launched the product with that feature and it became the biggest and most famous feature for the web app.

During my previous coop as an Analytics and Data Science Intern at TD, in the PBSA department, I developed a Recommendation System which would help us target customers with the Credit Card they are likely to buy and the reason for the purchase. I completely re-built feature engineering to increase the run-time efficiency by 200 times. This proved extremely valuable as it saved multiple days of wait to complete feature engineering. The code was written in a way that it could easily be moved either onto the cloud system (AKORA) or work locally if need be. I developed bash scripts which could be used to run the whole process at once instead of compiling and running each piece of code individually. This made it much easy to implement to code and run it.

The goal of all these stories was to show, who I am as a person. My intention at a coop or any job is not to just do my job but to help the team build better solutions and learn as much as possible while doing so. 

\end{cvletter}


%-------------------------------------------------------------------------------
% Print the signature and enclosures with above letter informations
\makeletterclosing

\end{document}
